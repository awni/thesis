\section{Datasets}
\label{arrhythmia:sec:data}

\subsection*{Training}
We collect and annotate a dataset of 64,121 ECG records from 29,163 patients.
The ECG data is sampled at a frequency of 200 Hz and is collected from a
single-lead, noninvasive and  continuous monitoring device called the Zio Patch
which has a wear period up to 14 days \cite{turakhia2013diagnostic}. Each ECG
record in the training set is 30 seconds long and can contain more than one
rhythm type. Each record is annotated by a clinical ECG expert: the expert
highlights segments of the signal and marks it as corresponding to one of the
14 rhythm classes.

The 30 second records were annotated using a web-based ECG annotation tool
designed for this work. Label annotations were done by a group of Certified
Cardiographic Technicians who have completed extensive training in arrhythmia
detection and a cardiographic certification examination by Cardiovascular
Credentialing International. The technicians were guided through the interface
before they could annotate records. All rhythms present in a strip were labeled
from their corresponding onset to offset, resulting in full segmentation of the
input ECG data. To improve labeling consistency among different annotators,
specific rules were devised regarding each rhythm transition.

We split the dataset into a training and validation set. The training set
contains 90\% of the data. We split the dataset so that there is no patient
overlap between the training and validation sets (as well as the test set
described below).

\subsection*{Testing}

We collect a test set of 336 records from 328 unique patients. For the test
set, ground truth annotations for each record were obtained by a committee of
three board-certified cardiologists; there are three committees responsible for
different splits of the test set. The cardiologists discussed each individual
record as a group and came to a consensus labeling. For each record in the test
set we also collect 6 individual annotations from cardiologists not
participating in the group. This is used to assess performance of the model
compared to an individual cardiologist.

\subsection*{Rhythm Classes}
We identify 12 heart arrhythmias, sinus rhythm and noise for a total of 14
output classes. The arrhythmias are characterized by a variety of features.
Table~\ref{tab:rhythms} in the Appendix shows an example of each rhythm type we
classify. The noise label is assigned when the device is disconnected from the
skin or when the baseline noise in the ECG makes identification of the
underlying rhythm impossible.

The morphology of the ECG during a single heart-beat as well as the pattern of
the activity of the heart over time determine the underlying rhythm. In some
cases the distinction between the rhythms can be subtle yet critical for
treatment. For example two forms of second degree AV Block, Mobitz I
(Wenckebach) and Mobitz II (here referred to as AVB\_TYPE2) can be difficult to
distinguish. Wenckebach is considered benign and Mobitz II is considered
pathological, requiring immediate attention \cite{dubin1996rapid}. 

Table~\ref{tab:rhythms} in the Appendix also shows the number of unique
patients in the training (including validation) set and test set for each
rhythm type.
