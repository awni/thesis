\section{Related Work}
\label{sec:arrhythmia:related}

Automatic high-accuracy methods for R-peak extraction have existed at least
since the mid 1980's \cite{pan1985real}. Current algorithms for R-peak
extraction tend to use wavelet transformations to compute features from the raw
ECG followed by finely-tuned threshold based classifiers \cite{li1995detection,
martinez2004wavelet}. Because accurate estimates of heart rate and heart rate
variability can be extracted from R-peak features, feature-engineered
algorithms are often used for coarse-grained heart rhythm classification,
including detecting tachycardias (fast heart rate), bradycardias (slow heart
rate), and irregular rhythms. However, such features alone are not sufficient
to distinguish between most heart arrhythmias since features based on the
atrial activity of the heart as well as other features pertaining to the QRS
morphology are needed.

Much work has been done to automate the extraction of other features from the
ECG. For example, beat classification is a common sub-problem of
heart-arrhythmia classification. Drawing inspiration from automatic speech
recognition, Hidden Markov models with Gaussian observation probability
distributions have been applied to the task of beat detection
\cite{coast1990approach}. Artificial neural networks have also been used for
the task of beat detection \cite{melo2000arrhythmia}. While these models have
achieved high-accuracy for some beat types, they are not yet sufficient for
high-accuracy heart arrhythmia classification and segmentation. For example,
\cite{artis1991detection} train a neural network to distinguish between Atrial
Fibrillation and Sinus Rhythm on the MIT-BIH dataset. While the network can
distinguish between these two classes with high-accuracy, it does not
generalize to noisier single-lead recordings or classify among the full range
of $15$ rhythms available in MIT-BIH. This is in part due to insufficient
training data, and because the model also discards critical information in the
feature extraction stage.

The most common dataset used to design and evaluate ECG algorithms is the
MIT-BIH arrhythmia database \cite{moody2001impact} which consists of 48
half-hour strips of ECG data. Other commonly used datasets include the MIT-BIH
Atrial Fibrillation dataset \cite{moody1983new} and the QT dataset
\cite{laguna1997database}. While useful benchmarks for R-peak extraction and
beat-level annotations, these datasets are too small for fine-grained
arrhythmia classification. The number of unique patients is in the single digit
hundreds or fewer for these benchmarks. A recently released dataset captured
from the AliveCor ECG monitor contains about 7000 records \cite{clifford2017}.
These records only have annotations for Atrial Fibrillation; all other
arrhythmias are grouped into a single bucket. The dataset we develop contains
29,163 unique patients and $14$ classes with hundreds of unique examples for
the rarest arrhythmias.

Machine learning models based on deep neural networks have consistently been
able to approach and often exceed human agreement rates when large annotated
datasets are available \cite{amodei2016deep, xiong2016achieving,he2015delving}.
These approaches have also proven to be effective in healthcare applications,
particularly in medical imaging where pretrained ImageNet models can be applied
\cite{esteva2017dermatologist, gulshan2016development}. We draw on work in
automatic speech recognition for processing time-series with deep convolutional
neural networks and recurrent neural networks \cite{hannun2014deepspeech,
sainath2013deep}, and techniques in deep learning to make the optimization of
these models tractable \cite{he2016deep, he2016identity, ioffe2015batch}.
