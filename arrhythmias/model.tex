\section{Model}
\label{sec:arrhythmia:model}

\subsection*{Problem Formulation}
The ECG arrhythmia detection task is a sequence-to-sequence task which takes as
input an ECG signal $X=[x_1,.. x_k]$, and outputs a sequence of labels $r=[r_1,
... r_n],$ such that each $r_i$ can take on one of $m$ different rhythm
classes. Each output label corresponds to a segment of the input. Together the
output labels cover the full sequence.

For a single example in the training set, we optimize the cross-entropy
objective function
\[
 \mathcal{L}(X, r) = \frac{1}{n} \sum_{i=1}^n \log p(R = r_i \mid X)
\]
where $p(\cdot)$ is the probability the network assigns to the $i$-th output
taking on the value $r_i$.

\subsection*{Model Architecture and Training}
We use a convolutional neural network for the sequence-to-sequence learning
task. The high-level architecture of the network is shown in
Figure~\ref{fig:net}. The network takes as input a time-series of raw ECG
signal, and outputs a sequence of label predictions. The 30 second long ECG
signal is sampled at 200Hz, and the model outputs a new prediction once every
second. We arrive at an architecture which is 33 layers of convolution followed
by a fully connected layer and a softmax. 

In order to make the optimization of such a network tractable, we employ
shortcut connections in a similar manner to those found in the Residual Network
architecture \cite{he2016identity}. The shortcut connections between
neural-network layers optimize training by allowing information to propagate
well in very deep neural networks. Before the input is fed into the network, it
is normalized using a robust normalization strategy. The network consists of
16 residual blocks with 2 convolutional layers per block. The convolutional
layers all have a filter length of 16 and have 64k filters, where $k$
starts out as 1 and is incremented every 4-th residual block. Every
alternate residual block subsamples its inputs by a factor of 2, thus the
original input is ultimately subsampled by a factor of $2^8$. When a residual
block subsamples the input, the corresponding shortcut connections also
subsample their input using a Max Pooling operation with the same subsample
factor. 

Before each convolutional layer we apply Batch Normalization
\cite{ioffe2015batch} and a rectified linear activation, adopting the
pre-activation block design \cite{he2016deep}. The first and last layers of the
network are special-cased due to this pre-activation block structure. We also
apply Dropout \cite{srivastava2014dropout} between the convolutional layers and
after the non-linearity. The final fully connected layer and softmax activation
produce a distribution over the 14 output classes for each time-step.

We train the networks from scratch, initializing the weights of the
convolutional layers as in \cite{he2015delving}. We use the Adam
\cite{kingma2014adam} optimizer with the default parameters and reduce the
learning rate by a factor of 10 when the validation loss stops improving. We
save the best model as evaluated on the validation set during the optimization
process.
