\section{Conclusion}

We develop a model which exceeds the cardiologist performance in detecting a
wide range of heart arrhythmias from single-lead ECG records. Key to the
performance of the model is a large annotated dataset and a very deep
convolutional network which can map a sequence of ECG samples to a sequence of
arrhythmia annotations. 

On the clinical side, future work should investigate extending the set of
arrhythmias and other forms of heart disease which can be automatically
detected with high-accuracy from single or multiple lead ECG records. For
example we do not detect Ventricular Flutter or Fibrillation. We also do not
detect Left or Right Ventricular Hypertrophy, Myocardial Infarction or a number
of other heart diseases which do not necessarily exhibit as arrhythmias. Some
of these may be difficult or even impossible to detect on a single-lead ECG but
can often be seen on a multiple-lead ECG.

Given that more than 300 million ECGs are recorded annually, high-accuracy
diagnosis from ECG can save expert clinicians and cardiologists considerable
time and decrease the number of misdiagnoses. Furthermore, we hope that this
technology coupled with low-cost ECG devices enables more widespread use of the
ECG as a diagnostic tool in places where access to a cardiologist is difficult.
