\section{Introduction}
\label{sec:arrhythmias:intro}

We develop a model which can diagnose irregular heart rhythms, also known as
arrhythmias, from single-lead ECG signals better than a cardiologist. Key to
exceeding expert performance is a deep convolutional network which can map a
sequence of ECG samples to a sequence of arrhythmia annotations along with a
novel dataset two orders of magnitude larger than previous datasets of its
kind.

Many heart diseases, including Myocardial Infarction, AV Block, Ventricular
Tachycardia and Atrial Fibrillation can all be diagnosed from ECG signals with
an estimated 300 million ECGs recorded annually~\cite{heden1996detection}. We
investigate the task of arrhythmia detection from the ECG record. This is known
to be a challenging task for computers but can usually be determined by an
expert from a single, well-placed lead.

Arrhythmia detection from ECG recordings is usually performed by expert
technicians and cardiologists given the high error rates of computerized
interpretation.  One study found that of all the computer predictions for
non-sinus rhythms, only about 50\% were correct~\cite{shah2007errors}; in
another study, only 1 out of every 7 presentations of second degree AV block
were correctly recognized by the algorithm~\cite{guglin2006common}. To
automatically detect heart arrhythmias in an ECG, an algorithm must implicitly
recognize the distinct wave types and discern the complex relationships between
them over time. This is difficult due to the variability in wave morphology
between patients as well as the presence of noise.

We train a 34-layer convolutional neural network (CNN) to detect arrhythmias in
arbitrary length ECG time-series. In addition to classifying noise and the
sinus rhythm, the network learns to classify and segment ten arrhythmia types
present in the time-series. The model is trained end-to-end on a single-lead
ECG signal sampled at 200Hz and a sequence of annotations for every second of
the ECG as supervision. To make the optimization of such a deep model
tractable, we use residual connections and
batch-normalization~\cite{he2016deep, ioffe2015batch}.  The depth increases
both the non-linearity of the computation as well as the size of the context
window for each classification decision.

We construct a dataset 500 times larger than other datasets of its
kind~\cite{moody2001impact, goldberger2000physiobank}. One of the most popular
previous datasets, the MIT-BIH corpus contains ECG recordings from 47 unique
patients. In contrast, we collect and annotate a dataset of about 54,000 unique
patients from a pool of nearly 300,000 patients who have used the Zio Patch
monitor\footnote[1]{iRhythm Technologies, San Francisco,
California}~\cite{turakhia2013diagnostic}. We intentionally select patients
exhibiting abnormal rhythms in order to make the class balance of the dataset
more even and thus the likelihood of observing unusual heart-activity high.

We test our model against board-certified cardiologists. A committee of three
cardiologists serve as gold-standard annotators for the 336 examples in the
test set. Our model achieves an AUC of greater than 0.90 for each of the 12
rhythm diagnoses, with an average AUC of 0.97. The model met or exceeded the
averaged cardiologist performance for all diagnoses. Average F1 scores for the
model (0.81) exceeded those for averaged cardiologists (0.78).
