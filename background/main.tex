\chapter{Background}

\section{Deep Neural Networks}

\subsection{Convolutional Neural Networks}
\subsection{Recurrent Neural Networks}
\subsection{Optimization}

\input{background/ctc.tex}

\section{Bibliographic Notes}

Convolutional neural networks were popularized following the seminal work of
Lecun, et al. in 1998~\cite{lecun1998gradient}. The LSTM RNN was proposed in
1997~\cite{hochreiter1997} and the GRU was proposed in 2014~\cite{cho2014}.
Olah gives a great introduction to RNNs and the LSTM cell~\cite{olah2015}.
Gradient clipping was first proposed by Pascanu, et al. in
2013~\cite{pascanu2013}. 

The CTC algorithm was first published by Graves et al. in
2006~\cite{graves2006}. The first experiments were on TIMIT, a popular phoneme
recognition benchmark~\cite{lopes2011}. Chapter 7 of Graves'
thesis~\cite{graves2012} also gives a detailed treatment of CTC.

One of the first applications of CTC to large vocabulary speech recognition was
by Graves et al. in 2014~\cite{graves2014}. They combined a hybrid DNN-HMM and
a CTC trained model to achieve state-of-the-art results. Hannun et al.
subsequently demonstrated state-of-the-art CTC based speech recognition on
larger benchmarks~\cite{hannun2014deepspeech}. A CTC model outperformed other
methods on an online handwriting recognition benchmark in
2007~\cite{liwicki2007}.

CTC has been used successfully in many other problems. Some examples are
lip-reading from video~\cite{assael2016}, action recognition from
video~\cite{huang2016} and keyword detection in audio~\cite{fernandez2007,
lengerich2016}.

Many extensions and improvements to CTC have been proposed. Here are a few.
The {\it Sequence Transducer} discards the conditional independence assumption
made by CTC~\cite{graves2012transducer}. As a consequence, the model allows the
output to be longer than the input. The {\it Gram-CTC} model generalizes CTC to
marginalize over n-gram output classes~\cite{liu2017}. Other works have
generalized CTC or proposed similar algorithms to account for segmental
structure in the output~\cite{wang2017, kong2016}.

The Hidden Markov Model was developed in the 1960's with the first application
to speech recognition in the 1970's. For an introduction to the HMM and
applications to speech recognition see Rabiner's canonical
tutorial~\cite{rabiner1989}.

Encoder-decoder models were developed in 2014~\cite{cho2014, sutskever2014}.
The online publication {\it Distill} has an in-depth guide to attention in
encoder-decoder models~\cite{olah2016}.
