\chapter{Introduction}

-TODO Write the intro

\section{Contributions per Chapter}

We present the following contributions in the chapters of this dissertation:

\begin{itemize}
    \setlength{\itemindent}{3.4em}
    \item[{\bf Chapter 2:}]{\bf Background.} This chapter provides some
        background on the models used throughout this work including RNNs and
        CTC. We start with a brief introduction to RNNs including LSTMs and
        GRUs. We also discuss various optimization algorithms used for RNNs as
        well as gradient clipping. We follow this with an in-depth look at CTC.
        We explain the CTC loss function and how to efficiently compute it as
        well as how to perform inference with a CTC trained model. The next
        section discusses properties and limitations of CTC and we follow this
        by placing CTC in context with HMMs and encoder-decoder models. We also
        include a short section on practical advice when using CTC. At the end
        of the background section we give a short bibliography and references
        for further study of the related material. 

    \item [{\bf Chapter 3:}]{\bf Application to Arrhythmia Detection.} In
        Chapter 3 we show how to apply deep convolutional networks to the task
        of arrhythmia detection. We collect a dataset substantially larger than
        previously studied corpora and with it develop a model which achieves
        excellent results. The model contains 34 layers, and in order to make
        such a deep model tractable we use techniques like batch normalization
        and residual connections. For ten arrhythmias, Sinus rhythm and noise
        the model achieves performance comparable to that of expert
        cardiologists when using a committee consensus label as the ground
        truth.

    \item [{\bf Chapter 4:}]{\bf First-pass Speech Recognition} Here we present
        a method to perform first-pass large vocabulary continuous speech
        recognition using only a neural network and language model. Deep neural
        network acoustic models are now commonplace in HMM-based speech
        recognition systems, but building such systems is a complex,
        domain-specific task. Prior work demonstrated the feasibility of
        discarding the HMM sequence modeling framework by directly predicting
        transcript text from audio. This chapter extends this approach in two
        ways. First, we demonstrate that a straightforward recurrent neural
        network architecture can achieve a high level of accuracy. Second, we
        propose and evaluate a modified prefix-search decoding algorithm. This
        approach to decoding enables first-pass speech recognition with a
        language model, completely unaided by the cumbersome infrastructure of
        HMM-based systems. Experiments on the Wall Street Journal corpus
        demonstrate fairly competitive word error rates, and the importance of
        bi-directional network recurrence. 

    \item [{\bf Chapter 5:}]{\bf Deep Speech: Scaling Up End-to-end Speech
        Recognition} In Chapter 5, we present a state-of-the-art speech
        recognition system developed using end-to-end deep learning. Our
        architecture is significantly simpler than traditional speech systems,
        which rely on laboriously engineered processing pipelines; these
        traditional systems also tend to perform poorly when used in noisy
        environments. In contrast, our system does not need hand-designed
        components to model background noise, reverberation, or speaker
        variation, but instead directly learns a function that is robust to
        such effects. We do not need a phoneme dictionary, nor even the concept
        of a "phoneme." Key to our approach is a well-optimized RNN training
        system that uses multiple GPUs, as well as a set of novel data
        synthesis techniques that allow us to efficiently obtain a large amount
        of varied data for training. Our system, called Deep Speech,
        outperforms previously published results on the widely studied
        Switchboard Hub5'00, achieving 16.0\% error on the full test set. Deep
        Speech also handles challenging noisy environments better than widely
        used, state-of-the-art commercial speech systems.

    \item [{\bf Chapter 6:}]{\bf Deep Speech 2: End-to-End Speech Recognition
        in English and Mandarin} We show that an end-to-end deep learning
        approach can be used to recognize either English or Mandarin Chinese
        speech–two vastly different languages. We show how to improve upon the
        Deep Speech model along three axis: the amount of annotated data, the
        architecture of the model and the efficiency of computation.  Th
        application of HPC techniques, enables experiments that previously took
        weeks to now run in days. This allows us to iterate more quickly to
        identify superior architectures and algorithms.  As a result, in
        several cases, our system is competitive with the transcription of
        human workers when benchmarked on standard datasets.  Finally, using a
        technique called Batch Dispatch with GPUs in the data center, we show
        that our system can be inexpensively deployed in an online setting,
        delivering low latency when serving users at scale.

    \item [{\bf Chapter 7:}]{\bf Keyword Spotting and Voice Activity Detection}
        In this chapter we show how a single neural network architecture can be
        used for two tasks: on-line keyword spotting and voice activity
        detection. We develop novel inference algorithms for an end-to-end RNN
        trained with the CTC loss function which allow our model to achieve
        high accuracy on both keyword spotting and voice activity detection
        without retraining. In contrast to prior voice activity detection
        models, this architecture does not require aligned training data and
        uses the same parameters as the keyword spotting model. This allows us
        to deploy a high quality voice activity detector with no additional
        memory or maintenance requirements.

\end{itemize}

\section{First Published Appearances}
Most contributions described in this thesis have first appeared in various
publications. Much of the background material on CTC in Chapter 2 first
appeared in~\cite{rajpurkar2017}. The results of Chapter 3 first appeared in
~\cite{hannun2017sequence}. The results on first-pass speech recognition in
Chapter 4 were in~\cite{hannun2014firstpass}. Chapters 5 on Deep Speech and 6
on Deep Speech 2 were previously published in~\cite{hannun2014deepspeech}
and~\cite{amodei2016deep}. Finally, Chapter 7 on keyword Spotting and voice
acitivity detection was first appeared in~\cite{lengerich2016}.
