\section{Experiments}
\label{sec:deepspeech:experiments}

We performed two sets of experiments to evaluate our system. In both cases we
use the model described in Section~\ref{sec:deepspeech:model} trained from a
selection of the datasets in Table~\ref{table:deepspeech:datasets} to predict
character-level transcriptions. The predicted probability vectors and language
model are then fed into our decoder to yield a word-level transcription, which
is compared with the ground truth transcription to yield the word error rate
(WER).

\subsection{Conversational speech: Switchboard Hub5'00 (full)}

To compare our system to prior research we use an accepted but highly
challenging test set, Hub5'00 (LDC2002S23). Some researchers split this set
into ``easy'' (Switchboard) and ``hard'' (CallHome) instances, often reporting
new results on the easier portion alone. We use the full set, which is the most
challenging case and report the overall word error rate.

We evaluate our system trained on only the 300 hour Switchboard conversational
telephone speech dataset and trained on both Switchboard (SWB) and Fisher
(FSH)~\cite{cieri2004fisher}, a 2000 hour corpus collected in a similar manner
as Switchboard. Many researchers evaluate models trained only with 300 hours
from Switchboard conversational telephone speech when testing on Hub5'00. In
part this is because training on the full 2000 hour Fisher corpus is
computationally difficult. Using the techniques mentioned in
Section~\ref{sec:deepspeech:optimization} our system is able perform a full
pass over the 2300 hours of data in just a few hours.

Since the Switchboard and Fisher corpora are distributed at a sample rate of
8kHz, we compute spectrograms of 80 linearly spaced log filter banks and an
energy term.  The filter banks are computed over windows of 20ms strided by
10ms. We did not evaluate more sophisticated features such as the mel-scale log
filter banks or the mel-frequency cepstral coefficients.

Speaker adaptation is critical to the success of current ASR
systems~\cite{vesely2013, sainath2013deep}, particularly when trained on 300
hour Switchboard. For the models we test on Hub5'00, we apply a simple form of
speaker adaptation by normalizing the spectral features on a per speaker basis.
Other than this, we do not modify the input features in any way.

For decoding, we use a 4-gram language model with a 30,000 word vocabulary
trained on the Fisher and Switchboard transcriptions. Again, hyperparameters
for the decoding objective are chosen via cross-validation on a held-out
development set.

The Deep Speech SWB model is a network of 5 hidden layers each with 2048
neurons trained on only 300 hour switchboard. The Deep Speech SWB + FSH model
is an ensemble of 4 RNNs each with 5 hidden layers of 2304 neurons trained on
the full 2300 hour combined corpus. All networks are trained on inputs of +/- 9
frames of context.

We report results in Table~\ref{table:deepspeech:hub5}. The model from Vesely
et al.  (DNN-GMM sMBR)~\cite{vesely2013} uses a sequence based loss function on
top of a DNN after using a typical hybrid DNN-HMM system to realign the
training set.  The performance of this model on the combined Hub5'00 test set
is the best previously published result. When trained on the combined 2300
hours of data the Deep Speech system improves upon this baseline by 2.4
absolute WER and 13.0\% relative. The model from Maas et al. (DNN-HMM
FSH)~\cite{maas2014} achieves 19.9 WER when trained on the Fisher 2000 hour
corpus. That system was built using Kaldi~\cite{povey2011}, state-of-the-art
open source speech recognition software. We include this result to demonstrate
that Deep Speech, when trained on a comparable amount of data is competitive
with the best existing ASR systems.

\begin{table}[ht!]
\centering
\begin{tabular}{l  c  c  c }
\toprule
Model & SWB & CH & Full \\
\midrule
Vesely et al. (GMM-HMM BMMI)~\cite{vesely2013}   & 18.6 & 33.0 & 25.8 \\
Vesely et al. (DNN-HMM sMBR)~\cite{vesely2013}    & 12.6 & 24.1  & 18.4 \\
Maas et al. (DNN-HMM SWB)~\cite{maas2014}  & 14.6 & 26.3  & 20.5 \\
Maas et al. (DNN-HMM FSH)~\cite{maas2014}  & 16.0 & 23.7  & 19.9 \\
Seide et al. (CD-DNN)~\cite{seide2011}     & 16.1 & n/a & n/a \\
Kingsbury et al. (DNN-HMM sMBR HF)~\cite{kingsbury2012}  & 13.3 & n/a & n/a \\
Sainath et al. (CNN-HMM)~\cite{sainath2013deep} & 11.5 & n/a & n/a \\
Soltau et al. (MLP/CNN+I-Vector)~\cite{soltau2014} & {\bf 10.4 } & n/a & n/a \\
{\bf Deep Speech SWB} & 20.0 & 31.8 & 25.9 \\
{\bf Deep Speech SWB + FSH} & 12.6 & {\bf 19.3} & {\bf 16.0} \\
\bottomrule
\end{tabular}
\caption{Published error rates (WER) on Switchboard dataset splits. The columns
    labeled ``SWB'' and ``CH'' are respectively the easy and hard subsets of
    Hub5'00.}
\label{table:deepspeech:hub5}
\end{table}

\subsection{Noisy speech}
\label{sec:deepspeech:expnoise}

Few standards exist for testing noisy speech performance, so we constructed our
own evaluation set of 100 noisy and 100 noise-free utterances from 10 speakers.
The noise environments included a background radio or TV; washing dishes in a
sink; a crowded cafeteria; a restaurant; and inside a car driving in the rain.
The utterance text came primarily from web search queries and text messages, as
well as news clippings, phone conversations, Internet comments, public
speeches, and movie scripts. We did not have precise control over the
signal-to-noise ratio (SNR) of the noisy samples, but we aimed for an SNR
between 2 and 6 dB. 

For the following experiments, we train our RNNs on all the datasets (more than
7000 hours) listed in Table~\ref{table:deepspeech:datasets}. Since we train for
15 to 20 epochs with newly synthesized noise in each pass, our model learns
from over 100,000 hours of novel data. We use an ensemble of 6 networks each
with 5 hidden layers of 2560 neurons. No form of speaker adaptation is applied
to the training or evaluation sets. We normalize training examples on a per
utterance basis in order to make the total power of each example consistent.
The features are 160 linearly spaced log filter banks computed over windows of
20ms strided by 10ms and an energy term. Audio files are resampled to 16kHz
prior to the featurization. Finally, from each frequency bin we remove the
global mean over the training set and divide by the global standard deviation,
primarily so the inputs are well scaled during the early stages of training.

As described in Section~\ref{sec:deepspeech:languagemodel}, we use a 5-gram
language model for the decoding. We train the language model on 220 million
phrases of the Common Crawl\footnote{commoncrawl.org}, selected such that at
least 95\% of the characters of each phrase are in the alphabet. Only the most
common 495,000 words are kept, the rest remapped to an \texttt{UNKNOWN} token.

We compared the Deep Speech system to several commercial speech systems: (1)
wit.ai, (2) Google Speech API, (3) Bing Speech and (4) Apple
Dictation.\footnote{wit.ai and Google Speech each have HTTP-based APIs. To test
Apple Dictation and Bing Speech, we used a kernel extension to loop audio
output back to audio input in conjunction with the OS X Dictation service and
the Windows 8 Bing speech recognition API.}

Our test is designed to benchmark performance in noisy environments. This
situation creates challenges for evaluating the web speech APIs: these systems
will give no result at all when the SNR is too low or in some cases when the
utterance is too long. Therefore we restrict our comparison to the subset of
utterances for which all systems returned a non-empty result.\footnote{This
leads to much higher accuracies than would be reported if we attributed 100\%
error in cases where an API failed to respond.} The results of evaluating each
system on our test files appear in Table~\ref{table:deepspeech:originalaudio}.  

To evaluate the efficacy of the noise synthesis techniques described in
Section~\ref{sec:deepspeech:noisesynth}, we trained two RNNs, one on 5000 hours
of raw data and the other trained on the same 5000 hours plus noise. On the 100
clean utterances both models perform about the same, 9.2 WER and 9.0 WER for
the clean trained model and the noise trained model respectively. However, on
the 100 noisy utterances the noisy model achieves 22.6 WER over the clean
model's 28.7 WER, a 6.1 absolute and 21.3\% relative improvement.

\begin{table}[ht!]
\centering
\begin{tabular}{l  c  c  c}
\toprule
System      &  Clean (94) &  Noisy (82) & Combined (176) \\
\midrule
Apple Dictation  & 14.24   & 43.76  & 26.73 \\
Bing Speech      &  11.73     &   36.12   &  22.05   \\
Google API       &  6.64   & 30.47  & 16.72 \\
wit.ai           &  7.94   & 35.06  & 19.41 \\
{\bf Deep Speech}       &  {\bf 6.56}   & {\bf 19.06}  & {\bf 11.85} \\
\bottomrule
\end{tabular}
\caption{Results (WER) for 5 systems evaluated on the original audio. Scores
         are reported {\it only} for utterances with predictions given by all
         systems. The number in parentheses next to each dataset, e.g. Clean (94),
         is the number of utterances scored.}
\label{table:deepspeech:originalaudio}
\end{table}



