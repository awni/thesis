\section{Introduction}
\label{sec:deepspeech:introduction}

Top speech recognition systems rely on sophisticated pipelines composed of
multiple algorithms and hand-engineered processing stages. In this paper, we
describe an end-to-end speech system, called ``Deep Speech'', where deep
learning supersedes these processing stages. Combined with a language model,
this approach achieves higher performance than traditional methods on hard
speech recognition tasks while also being much simpler. These results are made
possible by training a large recurrent neural network (RNN) using multiple GPUs
and thousands of hours of data. Because this system learns directly from data,
we do not require specialized components for speaker adaptation or noise
filtering. In fact, in settings where robustness to speaker variation and noise
are critical, our system excels: Deep Speech outperforms previously published
methods on the Switchboard Hub5'00 corpus, achieving 16.0 WER, and performs
better than commercial systems in noisy speech recognition tests.

Traditional speech systems use many heavily engineered processing stages,
including specialized input features, acoustic models, and Hidden Markov Models
(HMMs). To improve these pipelines, domain experts must invest a great deal of
effort tuning their features and models. The introduction of deep learning
algorithms~\cite{lee2009, mohamed2011, hinton2012, dahl2011a} has improved
speech system performance, usually by improving acoustic models.  While this
improvement has been significant, deep learning still plays only a limited role
in traditional speech pipelines. As a result, to improve performance on a task
such as recognizing speech in a noisy environment, one must laboriously
engineer the rest of the system for robustness. In contrast, our system applies
deep learning end-to-end using recurrent neural networks.  We take advantage of
the capacity provided by deep learning systems to learn from large datasets to
improve our overall performance. Our model is trained end-to-end to produce
transcriptions and thus, with sufficient data and computing power, can learn
robustness to noise or speaker variation on its own.

Tapping the benefits of end-to-end deep learning, however, poses several
challenges: (i) we must find innovative ways to build large, labeled training
sets and (ii) we must be able to train networks that are large enough to
effectively utilize all of this data. One challenge for handling labeled data
in speech systems is finding the alignment of text transcripts with input
speech. This problem has been addressed by Graves, Fern\'{a}ndez, Gomez and
Schmidhuber~\cite{graves2006}, thus enabling neural networks to easily consume
unaligned, transcribed audio during training. Meanwhile, rapid training of
large neural networks has been tackled by Coates et
al.~\cite{coates2013cotshpc}, demonstrating the speed advantages of multi-GPU
computation. We aim to leverage these insights to fulfill the vision of a
generic learning system, based on large speech datasets and scalable RNN
training, that can surpass more complicated traditional methods. This vision
is inspired partly by the work of Lee~et.~al.~\cite{lee2009} who
applied early unsupervised feature learning techniques to replace hand-built
speech features.

We have chosen our RNN model specifically to map well to GPUs and we use a
novel model partition scheme to improve parallelization. Additionally, we
propose a process for assembling large quantities of labeled speech data
exhibiting the distortions that our system should learn to handle. Using a
combination of collected and synthesized data, our system learns robustness to
realistic noise and speaker variation (including Lombard
Effect~\cite{junqua1993lombard}). Taken together, these ideas suffice to build
an end-to-end speech system that is at once simpler than traditional pipelines
yet also performs better on difficult speech tasks. Deep Speech achieves a word
error rate of 16.0 on the full Switchboard Hub5'00 test set---the best
published result. Further, on a new noisy speech recognition dataset of our own
construction, our system achieves a word error rate of 19.1 where the best
commercial systems achieve 30.5 error.
