\section{Related Work}
\label{sec:deepspeech:related}

Several parts of our work are inspired by previous results. Neural network
acoustic models and other connectionist approaches were first introduced to
speech pipelines in the early 1990s~\cite{bourlard93, renals1994, ellis1999}.
These systems, similar to DNN acoustic models~\cite{mohamed2011, hinton2012,
dahl2011a}, replace only one stage of the speech recognition pipeline.
Mechanically, our system is similar to other efforts to build end-to-end speech
systems from deep learning algorithms. For example,
Graves~et~al.~\cite{graves2006} have previously introduced the ``Connectionist
Temporal Classification'' (CTC) loss function for scoring transcriptions
produced by RNNs and, with LSTM networks, have previously applied this approach
to speech~\cite{graves2014}. We similarly adopt the CTC loss for part of our
training procedure but use much simpler recurrent networks with
rectified-linear activations~\cite{glorot2011, maas2013, nair2010}. Our
recurrent network is similar to the bidirectional RNN used by Hannun et
al.~\cite{hannun2014firstpass}, but with multiple changes to enhance its
scalability. By focusing on scalability, we have shown that these simpler
networks can be effective even without the more complex LSTM machinery.

Our work is certainly not the first to exploit scalability to improve
performance of DL algorithms. The value of scalability in deep learning is
well-studied~\cite{coates2011b, le2013} and the use of parallel processors
(including GPUs) has been instrumental to recent large-scale DL
results~\cite{szegedy2015, le2013}. Early ports of DL algorithms to GPUs
revealed significant speed gains~\cite{raina2009}. Researchers have also begun
choosing designs that map well to GPU hardware to gain even more efficiency,
including convolutional~\cite{krizhevsky2012imagenet, ciresan2011,
sainath2013b} and locally connected~\cite{coates2013cotshpc, ciresan2012}
networks, especially when optimized libraries like
cuDNN~\cite{chetlur2014cudnn} and BLAS are available. Indeed, using
high-performance computing infrastructure, it is possible today to train neural
networks with more than 10 billion connections~\cite{coates2013cotshpc} using
clusters of GPUs. These results inspired us to focus first on making scalable
design choices to efficiently utilize many GPUs before trying to engineer the
algorithms and models themselves.

With the potential to train large models, there is a need for large training
sets as well. In other fields, such as computer vision, large labeled training
sets have enabled significant leaps in performance as they are used to feed
larger and larger DL systems~\cite{szegedy2015, krizhevsky2012imagenet}. In
speech recognition, however, such large training sets are less common, with
typical benchmarks having training sets ranging from tens of hours (e.g. the
Wall Street Journal corpus with 80 hours) to several hundreds of hours (e.g.
Switchboard and Broadcast News). Larger benchmark datasets, such as the Fisher
corpus~\cite{cieri2004fisher} with 2000 hours of transcribed speech, are rare
and only recently being studied. To fully utilize the expressive power of the
recurrent networks available to us, we rely not only on large sets of labeled
utterances, but also on synthesis techniques to generate novel examples. This
approach is well known in computer vision~\cite{sapp2008, lecun2004,
coates2011} but we have found this especially convenient and effective for
speech when done properly.
