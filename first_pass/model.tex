\section{Model}
\label{sec:first_pass:model}

We train neural networks using the CTC loss function to do maximum
likelihood training of letter sequences given acoustic features as
input. We consider a single utterance as a training example consisting
of an acoustic feature matrix $X$ and word transcription $Y$. The CTC
objective function maximizes the log probability $\log p(Y \mid X)$. We
reserve a full exposition of the loss function here because our
formulation follows exactly the previous work on using CTC to predict
the characters of an utterance transcription \cite{graves2014, graves2006}. 

\subsection{Deep Neural Networks}
\label{sec:first_pass:model:dnn}

With the loss function fixed we must next define how we compute $p(c \mid
x_t)$, the predicted distribution over output characters $c$ given the audio
features $x_t$ at time $t$. While many function approximators are possible for
this task, we choose as our most basic model a DNN. A DNN computes the
distribution $p(c \mid x_t)$ using a series of hidden layers followed by an output
layer. Given an input vector $x_t$ the first hidden layer activations are a
vector computed as,
\begin{align*}
  h^{(1)} = \sigma(W^{(1)T} x_t + b^{(1)}).
\end{align*}
The matrix $W^{(1)}$ and vector $b^{(1)}$ are the weight matrix and bias vector
for the layer. The function $\sigma(\cdot)$ is a point-wise nonlinearity. We
use rectifier nonlinearities and thus choose, $\sigma(z) = \max (z, 0)$.

DNNs can have arbitrarily many hidden layers. After the first hidden
layer, the hidden activations $h^{(i)}$ for layer $i$ are computed as,
\begin{align*}
  h^{(i)} = \sigma(W^{(i)T} h^{(i-1)} + b^{(i)}).
\end{align*}

To obtain a proper distribution over the set of possible characters $c$ the
final layer of the network is a \emph{softmax} output layer of the form,
\begin{align*}
  p(c=c_k \mid x_t) = \frac{\exp(- ( W^{(s)T}_k h^{(i-1)} + b^{(s)}_k))}
  {\sum_j \exp(- ( W^{(s)T}_j h^{(i-1)} + b^{(s)}_j))},
\end{align*}
where $W^{(s)}_k$ is the $k$'th column of the output weight matrix $W^{(s)}$
and $b^{(s)}_k$ is a scalar bias term. 

We can compute a subgradient for all parameters of the DNN given a training
example and thus utilize gradient-based optimization techniques. Note that this
same DNN formulation is commonly used in DNN-HMM models to predict a
distribution over senones instead of characters.

\subsection{Recurrent Deep Neural Networks}

A transcription $W$ has many temporal dependencies which a DNN may not
sufficiently capture. At each timestep $t$ the DNN computes its output using
only the input features $x_t$, ignoring previous hidden representations and
output distributions. To enable better modeling of the temporal dependencies
present in a problem, we use a RDNN. In a RDNN we select one hidden layer $j$
to have a temporally recurrent weight matrix $W^{(f)}$ and compute the layer's
hidden activations as,
\begin{align*}
  h^{(j)}_t = \sigma(W^{(j)T} h^{(j-1)}_t +  W^{(f)T} h^{(j)}_{t-1} + b^{(j)}).
\end{align*}
Note that we now make the distinction $h^{(j)}_t$ for the hidden activation
vector of layer $j$ at timestep $t$ since it now depends upon the activation
vector of layer $j$ at time $t-1$.

When working with RDNNs, we found it important to use a modified version of the
rectifier nonlinearity. This modified function selects $\sigma(z) = \min( \max
(z, 0), 20)$ which clips large activations to prevent divergence during network
training. Setting the maximum allowed activation to 20 results in the clipped
rectifier acting as a normal rectifier function in all but the most extreme
cases.

Aside from these changes, computations for a RDNN are the same as those in a
DNN as described in \ref{sec:first_pass:model:dnn}. Like the DNN, we can
compute a subgradient for a RDNN using a method sometimes called
backpropagation through time. In our experiments we always compute the gradient
completely through time rather than truncating to obtain an approximate
subgradient.

\subsection{Bi-Directional Recurrent Deep Neural Networks}

While forward recurrent connections reflect the temporal nature of the audio
input, a perhaps more powerful sequence transduction model is a BRDNN, which
maintains state both forwards and backwards in time. Such a model can integrate
information from the entire temporal extent of the input features when making
each prediction. We extend the RDNN to form a BRDNN by again choosing a
temporally recurrent layer $j$. The BRDNN creates both a forward and backward
intermediate hidden representation which we call $h_t^{(f)}$ and $h_t^{(b)}$
respectively. 

We use the temporal weight matrices $W^{(f)}$ and $W^{(b)}$ to propagate
$h_t^{(f)}$ forward in time and $h_t^{(b)}$ backward in time respectively. We
update the forward and backward components via the equations,
\begin{align*}
    h^{(f)}_t &= \sigma(W^{(j)T} h^{(j-1)}_t +  W^{(f)T} h^{(f)}_{t-1} + b^{(j)}), \\
    h^{(b)}_t &= \sigma(W^{(j)T} h^{(j-1)}_t +  W^{(b)T} h^{(b)}_{t+1} + b^{(j)}).
\end{align*}

Note that the recurrent forward and backward hidden representations are
computed entirely independently from each other. As with the RDNN we use the
modified nonlinearity function $\sigma(z) = \min( \max (z,0), 20)$. To obtain
the final representation $h^{(j)}_t$ for the layer we sum the two temporally
recurrent components,
\begin{align}
\begin{split}
  h^{(j)}_t = h^{(f)}_t + h^{(b)}_t.
\end{split}
\end{align}

Aside from this change to the recurrent layer the BRDNN computes its output
using the same equations as the RDNN. As for other models, we can compute a
subgradient for the BRDNN directly to perform gradient-based optimization.
