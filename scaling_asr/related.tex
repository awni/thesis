\section{Related Work}
 \label{sec:scaling_asr:related}

This work is inspired by previous work in both deep learning and speech
recognition. Feed-forward neural network acoustic models were explored more
than 20 years ago~\cite{bourlard93, renals1994, ellis1999}. Recurrent neural
networks and networks with convolution were also used in speech recognition
around the same time~\cite{robinson1996, waibel1989}. More recently DNNs have
become a fixture in the ASR pipeline with almost all state of the art speech
work containing some form of deep neural network~\cite{mohamed2011, hinton2012,
dahl2011a, dahl2011, jaitly2012, seide2011}. Convolutional networks have also
been found beneficial for acoustic models~\cite{abdelhamid2012,
sainath2013deep}. Recurrent neural networks, typically LSTMs, are just
beginning to be deployed in state-of-the art recognizers~\cite{graves2013,
sak2014, sak2014b} and work well together with convolutional layers for the
feature extraction~\cite{sainath2015}. Models with both
bidirectional~\cite{graves2013} and unidirectional recurrence have been
explored as well.

End-to-end speech recognition is an active area of research, showing compelling
results when used to re-score the outputs of a DNN-HMM~\cite{graves2014} and
standalone~\cite{hannun2014deepspeech}. Two methods are currently used to map
variable length audio sequences directly to variable length transcriptions. The
RNN encoder-decoder paradigm uses an encoder RNN to map the input to a fixed
length vector and a decoder network to expand the fixed length vector into a
sequence of output predictions~\cite{cho2014, sutskever2014}. Adding an
attentional mechanism to the decoder greatly improves performance of the
system, particularly with long inputs or outputs~\cite{bahdanau2014}. In
speech, the RNN encoder-decoder with attention performs well both in predicting
phonemes~\cite{chorowski2014} or graphemes~\cite{bahdanau2016, chan2016}. 

The other commonly used technique for mapping variable length audio input to
variable length output is the CTC loss function~\cite{graves2006} coupled with
an RNN to model temporal information. The CTC-RNN model performs well in
end-to-end speech recognition with grapheme outputs~\cite{graves2014,
hannun2014firstpass, hannun2014deepspeech, maas2015}. The CTC-RNN model has
also been shown to work well in predicting phonemes~\cite{miao2015, sak2015},
though a lexicon is still needed in this case.  Furthermore it has been
necessary to pre-train the CTC-RNN network with a DNN cross-entropy network
that is fed frame-wise alignments from a GMM-HMM system~\cite{sak2015}. In
contrast, we train the CTC-RNN networks from scratch without the need of
frame-wise alignments for pre-training.

Exploiting scale in deep learning has been central to the success of the field
thus far~\cite{krizhevsky2012imagenet, le2013}. Training on a single GPU
resulted in substantial performance gains~\cite{raina2009}, which were
subsequently scaled linearly to two~\cite{krizhevsky2012imagenet} or more
GPUs~\cite{coates2013cotshpc}. We take advantage of work in increasing
individual GPU efficiency for low-level deep learning
primitives~\cite{chetlur14}. We build on the past work in using
model-parallelism~\cite{coates2013cotshpc}, data-parallelism~\cite{dean2012} or
a combination of the two~\cite{szegedy2015, hannun2014deepspeech} to create a
fast and highly scalable system for training deep RNNs in speech recognition.

Data has also been central to the success of end-to-end speech recognition,
with over 7000 hours of labeled speech used in Deep Speech 1
(DS1)~\cite{hannun2014deepspeech}. Data augmentation has been highly effective
in improving the performance of deep learning in computer
vision~\cite{lecun2004, sapp2008, coates2011}. This has also been shown to
improve speech systems~\cite{gales2009, hannun2014deepspeech}. Techniques used
for data augmentation in speech range from simple noise
addition~\cite{hannun2014deepspeech} to complex perturbations such as
simulating changes to the vocal tract length and rate of speech of the
speaker~\cite{jaitly2013, ko2015}.

Existing speech systems can also be used to bootstrap new data collection. In
one approach, the authors use one speech engine to align and filter a thousand
hours of read speech~\cite{panayotov2015}. In another approach, a heavy-weight
offline speech recognizer is used to generate transcriptions for tens of
thousands of hours of speech~\cite{kapralova2014}. This is then passed through
a filter and used to re-train the recognizer, resulting in significant
performance gains. We draw inspiration from these past approaches in
bootstrapping larger datasets and data augmentation to increase the effective
amount of labeled data for our system.
