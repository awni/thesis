\section{Introduction}
\label{sec:scaling_asr:introduction}

Decades worth of hand-engineered domain knowledge has gone into current
state-of-the-art automatic speech recognition (ASR) pipelines.  A simple but
powerful alternative solution is to train such ASR models end-to-end, using
deep learning to replace most modules with a single
model~\cite{hannun2014deepspeech}. We present the second generation of our
speech system that exemplifies the major advantages of end-to-end learning. The
{Deep Speech 2} ASR pipeline approaches or exceeds the accuracy of Amazon
Mechanical Turk human workers on several benchmarks, works in multiple
languages with little modification, and is deployable in a production setting.
It thus represents a significant step towards a single ASR system that
addresses the entire range of speech recognition contexts handled by humans.
Since our system is built on end-to-end deep learning, we can employ a spectrum
of deep learning techniques: capturing large training sets, training larger
models with high performance computing, and methodically exploring the space of
neural network architectures. We show that through these techniques we are able
to reduce error rates of our previous end-to-end
system~\cite{hannun2014deepspeech} in English by up to 43\%, and can also
recognize Mandarin speech with high accuracy.

One of the challenges of speech recognition is the wide range of variability in
speech and acoustics. As a result, modern ASR pipelines are made up of numerous
components including complex feature extraction, acoustic models, language and
pronunciation models, speaker adaptation, etc. Building and tuning these
individual components makes developing a new speech recognizer very hard,
especially for a new language.  Indeed, many parts do not generalize well
across environments or languages and it is often necessary to support multiple
application-specific systems in order to provide acceptable accuracy. This
state of affairs is different from human speech recognition: people have the
innate ability to learn any language during childhood, using general skills to
learn language. After learning to read and write, most humans can transcribe
speech with robustness to variation in environment, speaker accent and noise,
without additional training for the transcription task.  To meet the
expectations of speech recognition users, we believe that a single engine must
learn to be similarly competent; able to handle most applications with only
minor modifications and able to learn new languages from scratch without
dramatic changes.  Our end-to-end system puts this goal within reach, allowing
us to approach or exceed the performance of human workers on several tests in
two very different languages: Mandarin and English.

Since {Deep Speech 2} (DS2) is an end-to-end deep learning system, we can
achieve performance gains by focusing on three crucial components:  the model
architecture, large labeled training datasets, and computational scale. This
approach has also yielded great advances in other application areas such as
computer vision and natural language.  This paper details our contribution to
these three areas for speech recognition, including an extensive investigation
of model architectures and the effect of data and model size on recognition
performance.  In particular, we describe numerous experiments with neural
networks trained with the Connectionist Temporal Classification (CTC) loss
function~\cite{graves2006} to predict speech transcriptions from audio.  We
consider networks composed of many layers of recurrent connections,
convolutional filters, and nonlinearities, as well as the impact of a specific
instance of Batch Normalization~\cite{ioffe2015batch} (BatchNorm) applied to
RNNs.  We not only find networks that produce much better predictions than
those in previous work~\cite{hannun2014deepspeech}, but also find instances of
recurrent models that can be deployed in a production setting with no
significant loss in accuracy.

Beyond the search for better model architecture, deep learning systems benefit
greatly from large quantities of training data.  We detail our data capturing
pipeline that has enabled us to create larger datasets than what is typically
used to train speech recognition systems.  Our English speech system is trained
on 11,940 hours of speech, while the Mandarin system is trained on 9,400 hours.
We use data synthesis to further augment the data during training.

Training on large quantities of data usually requires the use of larger models.
Indeed, our models have many more parameters than those used in our previous
system. Training a single model at these scales requires tens of
exaFLOPs\footnote{1 exaFLOP = $10^{18}$ FLoating-point OPerations.} that would
require 3-6 weeks to execute on a single GPU. This makes model exploration a
very time consuming exercise, so we have built a highly optimized training
system that uses 8 or 16 GPUs to train one model. In contrast to previous
large-scale training approaches that use parameter servers and asynchronous
updates~\cite{dean2012, chilimbi2014}, we use synchronous SGD, which is easier
to debug while testing new ideas, and also converges faster for the same degree
of data parallelism. To make the entire system efficient, we describe
optimizations for a single GPU as well as improvements to scalability for
multiple GPUs. We employ optimization techniques typically found in High
Performance Computing to improve scalability. These optimizations include a
fast implementation of the CTC loss function on the GPU, and a custom memory
allocator. We also use carefully integrated compute nodes and a custom
implementation of all-reduce to accelerate inter-GPU communication. Overall the
system sustains approximately 50 teraFLOP/second when training on 16 GPUs. This
amounts to 3 teraFLOP/second per GPU which is about 50\% of peak theoretical
performance. This scalability and efficiency cuts training times down to 3 to 5
days, allowing us to iterate more quickly on our models and datasets.

We benchmark our system on several publicly available test sets and compare the
results to our previous end-to-end system~\cite{hannun2014deepspeech}.  Our
goal is to eventually reach human-level performance not only on specific
benchmarks, where it is possible to improve through dataset-specific tuning,
but on a range of benchmarks that reflects a diverse set of scenarios. To that
end, we have also measured the performance of human workers on each benchmark
for comparison.  We find that our system outperforms humans in some
commonly-studied benchmarks and has significantly closed the gap in much harder
cases.  In addition to public benchmarks, we show the performance of our
Mandarin system on internal datasets that reflect real-world product scenarios.

Deep learning systems can be challenging to deploy at scale.  Large neural
networks are computationally expensive to evaluate for each user utterance, and
some network architectures are more easily deployed than others. Through model
exploration, we find high-accuracy, deployable network architectures, which we
detail here. We also employ a batching scheme suitable for GPU hardware called
Batch Dispatch that leads to an efficient, real-time implementation of our
Mandarin engine on production servers.  Our implementation achieves a 98th
percentile compute latency of 67 milliseconds, while the server is loaded with
10 simultaneous audio streams.
