\section{Data}
\label{sec:scaling_asr:data}

Large-scale deep learning systems require an abundance of labeled training
data. We have collected an extensive training dataset for both English and
Mandarin speech models, in addition to augmenting our training with publicly
available datasets. In English we use 11,940 hours of labeled speech data
containing 8 million utterances summarized in
Table~\ref{table:scaling_asr:englishdata}.  For the Mandarin system we use
9,400 hours of labeled audio containing 11 million utterances. The Mandarin
speech data consists of internal Baidu corpora, representing a mix of read
speech and spontaneous speech, in both standard Mandarin and accented Mandarin.

\begin{table}
\centering
\begin{tabular}{l l r}
 \toprule
 Dataset & Speech Type & Hours \\
 \midrule
 WSJ          & read           &   80  \\
 Switchboard  & conversational &  300  \\
 Fisher       & conversational & 2000  \\
 LibriSpeech  & read           &  960  \\
 Baidu        & read           & 5000  \\
 Baidu        & mixed          & 3600  \\
 \midrule
 Total       &                &  11940 \\
 \bottomrule
\end{tabular}
\caption{Summary of the datasets used to train DS2 in English. The Wall Street
        Journal (WSJ), Switchboard and Fisher~\cite{cieri2004fisher} corpora are
        all published by the Linguistic Data Consortium. The LibriSpeech
        dataset~\cite{panayotov2015} is available free on-line. The other datasets
        are internal Baidu corpora.}
\label{table:scaling_asr:englishdata}
\end{table}


\subsection{Dataset Construction}

Some of the internal English (3,600 hours) and Mandarin (1,400 hours) datasets
were created from raw data captured as long audio clips with noisy
transcriptions. The length of these clips ranged from several minutes to more
than hour, making it impractical to unroll them in time in the RNN during
training. To solve this problem, we developed an alignment, segmentation and
filtering pipeline that can generate a training set with shorter utterances and
few erroneous transcriptions.

The first step in the pipeline is to use an existing bidirectional RNN model
trained with CTC to align the transcription to the frames of audio. For a given
audio-transcript pair, $(x, y)$, we find the alignment that maximizes
\begin{equation*}
\ell^* = \argmax_{\ell \in \text{Align}(x,y)} \prod_t^T p_{\text{ctc}}(\ell_t | x; \theta).
\end{equation*}
This is essentially a Viterbi alignment found using a RNN model trained with
CTC. Since Equation~\ref{eq:scaling_asr:ctc} integrates over the alignment, the
CTC loss function is never explicitly asked to produce an accurate alignment.
In principle, CTC could choose to emit all the characters of the transcription
after some fixed delay and this can happen with unidirectional
RNNs~\cite{sak2015}. However, we found that CTC produces an accurate alignment
when trained with a bidirectional RNN.

Following the alignment is a segmentation step that splices the audio and the
corresponding aligned transcription whenever it encounters a long series of
consecutive \emph{blank} labels occurs, since this usually denotes a stretch of
silence. By tuning the number of consecutive \emph{blank}s, we can tune the
length of the utterances generated. For the English speech data, we also
require a \emph{space} token to be within the stretch of \emph{blank}s in order
to segment only on word boundaries. We tune the segmentation to generate
utterances that are on average 7 seconds long.

The final step in the pipeline removes erroneous examples that arise from a
failed alignment. We crowd source the ground truth transcriptions for several
thousand examples. The word level edit distance between the ground truth and
the aligned transcription is used to produce a \emph{good} or \emph{bad} label.
The threshold for the word level edit distance is chosen such that the
resulting WER of the \emph{good} portion of the development set is less than
5\%. We then train a linear classifier to accurately predict bad examples given
the input features generated from the speech recognizer. We find the following
features useful: the raw CTC cost, the CTC cost normalized by the sequence
length, the CTC cost normalized by the transcript length, the ratio of the
sequence length to the transcript length, the number of words in the
transcription and the number of characters in the transcription. For the
English dataset, we find that the filtering pipeline reduces the WER from 17\%
to 5\% while retaining more than 50\% of the examples.

\subsection{Data Augmentation}

We augment our training data by adding noise to increase the effective size of
our training data and to improve our robustness to noisy
speech~\cite{hannun2014deepspeech}. Although the training data contains some
intrinsic noise, we can increase the quantity and variety of noise through
augmentation. Too much noise augmentation tends to make optimization difficult
and can lead to worse results, and too little noise augmentation makes the
system less robust to low signal-to-noise speech.  We find that a good balance
is to add noise to 40\% of the utterances that are chosen at random. The noise
source consists of several thousand hours of randomly selected audio clips
combined to produce hundreds of hours of noise.

\subsection{Scaling Data}

Our English and Mandarin corpora are substantially larger than those commonly
reported in speech recognition literature. In
Table~\ref{table:scaling_asr:datascale}, we show the effect of increasing the
amount of labeled training data on WER. This is done by randomly sampling the
full dataset before training. For each dataset, the model was trained for up to
20 epochs though usually early-stopped based on the error on a held out
development set. We note that the WER decreases with a power law for both the
regular and noisy development sets. The WER decreases by $\sim$40\% relative
for each factor of 10 increase in training set size. We also observe a
consistent gap in WER ($\sim$60\% relative) between the regular and noisy
datasets, implying that more data benefits both cases equally. 

This implies that a speech system will continue to improve with more labeled
training data. We hypothesize that equally as important as increasing raw
number of hours is increasing the number of speech \emph{contexts} that are
captured in the dataset. A context can be any property that makes speech unique
including different speakers, background noise, environment, and microphone
hardware. While we do not have the labels needed to validate this claim, we
suspect that measuring WER as a function of speakers in the dataset would lead
to much larger relative gains than simple random sampling.

\begin{table}
\centering
\begin{tabular}{r r r  r  r r r  r r r}
\toprule
\multicolumn{3}{c}{Fraction of Data} & Hours & \multicolumn{3}{c}{Regular Dev} & \multicolumn{3}{c}{Noisy Dev} \\
\midrule
& 1\%   & & 120   & & 29.23 & & & 50.97 & \\
& 10\%  & & 1200  & & 13.80 & & & 22.99 & \\
& 20\%  & & 2400  & & 11.65 & & & 20.41 & \\
& 50\%  & & 6000  & & 9.51  & & & 15.90 & \\
& 100\% & & 12000 & & 8.46  & & & 13.59 & \\
\bottomrule
\end{tabular}
\caption{Comparison of English WER for Regular and Noisy development sets on
         increasing training dataset size. The architecture is a 9-layer model with
         2 layers of 2D-invariant convolution and 7 recurrent layers with 68M
         parameters.}
\label{table:scaling_asr:datascale}
\end{table}
